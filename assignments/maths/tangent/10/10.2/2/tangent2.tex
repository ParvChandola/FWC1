\documentclass[12pt]{article}
\usepackage{graphicx}
\usepackage{amsmath}
\usepackage{mathtools}
\usepackage{gensymb}
\usepackage{tabularx}
\usepackage{array}
\usepackage[latin1]{inputenc}
\usepackage{fullpage}
\usepackage{color}
\usepackage{array}
\usepackage{longtable}
\usepackage{calc}
\usepackage{multirow}
\usepackage{hhline}
\usepackage{ifthen}
\usepackage{lscape}
\usepackage{float}

\newcommand{\mydet}[1]{\ensuremath{\begin{vmatrix}#1\end{vmatrix}}}
\providecommand{\brak}[1]{\ensuremath{\left(#1\right)}}
\providecommand{\norm}[1]{\left\lVert#1\right\rVert}
\providecommand{\abs}[1]{\left\vert#1\right\vert}
\newcommand{\solution}{\noindent \textbf{Solution: }}
\newcommand{\myvec}[1]{\ensuremath{\begin{pmatrix}#1\end{pmatrix}}}
\let\vec\mathbf

\def\inputGnumericTable{}

\begin{document}
\begin{center}
\textbf\large{TANGENTS AND NORMALS}

\end{center}
\section*{Excercise 10.2}
Q2.In fig \ref{fig:Fig1}, if TP and TQ are two tangents to a circle with centre O so that $\angle{POQ} = 110\degree$ then $\angle{PTQ}$ is equal to.

\solution
Let us assume the centre $\vec{O} = \myvec{0\\0}$. Any point $\vec{X}$ on the circle is given as
\begin{align}
	\vec{X} = \vec{O}+\myvec{\cos\theta\\\sin\theta}
\end{align}
The input parameters are given as
\input{./table/table1.tex}
For tangent $TP$
\begin{align}
	\vec{n}_1 &= \vec{P}-\vec{O}\\
	&= \myvec{\cos{110}\degree\\\sin{110}\degree} =  \myvec{1\\\tan{110}\degree}\\
	\vec{m}_1 &= \myvec{1\\-\cot{110}\degree}
\end{align}
For tangent $TQ$
\begin{align}
	\vec{n}_2 &= \vec{Q}-\vec{O}\\
	&= \myvec{1\\0}\\
	\vec{m}_2 &= \myvec{0\\1}
\end{align}
The equation of $TP$ is given as
\begin{align}
	\vec{n}_1^\top\brak{\vec{x}-\myvec{\cos{110}\degree\\\sin{110}\degree}} &= 0\\
	\label{eq:eq1}
	\myvec{-0.342 & 0.939}\vec{x} &= 1
\end{align}
The equation of $TQ$ is given as
\begin{align}
	\vec{n}_2^\top\brak{\vec{x}-\myvec{1\\0}} &= 0\\
	\label{eq:eq2}
	\myvec{1&0}\vec{x} &= 1
\end{align}
The tangent point can be calculated by solving \eqref{eq:eq1} and \eqref{eq:eq2}
\begin{align}
	\myvec{-0.342&0.939\\1&0}\myvec{x\\y} &= \myvec{1\\1}\\
	\implies \myvec{x\\y} &= \myvec{1\\1.428}
\end{align}
The angle between two lines with slope $\vec{m}_1 \text{ and } \vec{m}_2$ s given as
\begin{align}
	\cos\theta &= \frac{\vec{m}_1^\top\vec{m}_2}{\norm{\vec{m}_1}\norm{\vec{m}_2}}\\
	&= \frac{\myvec{1&-\cot{110}\degree}\myvec{0\\1}}{\brak{\csc{110}\degree}\brak{1}}\\
	&= -\cos{110}\degree\\
	\implies \theta &= 70\degree
\end{align}
Hence, $\angle{PTQ} = 70\degree$. See Fig \ref{fig:Fig1}
\begin{figure}[!h]
	\begin{center} 
	    \includegraphics[width=\columnwidth]{figs/tangent2}
	\end{center}
\caption{}
\label{fig:Fig1}
\end{figure}

\end{document}

















