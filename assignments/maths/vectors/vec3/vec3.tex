
\documentclass[12pt]{article}
\usepackage{graphicx}
\usepackage{amsmath}
\usepackage{mathtools}
\usepackage{gensymb}

\newcommand{\mydet}[1]{\ensuremath{\begin{vmatrix}#1\end{vmatrix}}}
\providecommand{\brak}[1]{\ensuremath{\left(#1\right)}}
\providecommand{\norm}[1]{\left\lVert#1\right\rVert}
\newcommand{\solution}{\noindent \textbf{Solution: }}
\newcommand{\myvec}[1]{\ensuremath{\begin{pmatrix}#1\end{pmatrix}}}
\let\vec\mathbf

\begin{document}
\begin{center}
\textbf\large{CHAPTER-10 \\ VECTOR ALGEBRA}

\end{center}
\section*{Excercise 10.3}

Q10.If $\vec{a} = 2\hat{i}+2\hat{j}+3\hat{k}, \vec{b} = -\hat{i}+2\hat{j}+\hat{k} \text{ and } \vec{c} = 3\hat{i}+\hat{j}$ are such that $\vec{a}+\lambda \vec{b}$ is perpendicular to $\vec{c}$, then find the value of $\lambda$.

\solution 
\begin{align}
	\vec{a}+\lambda \vec{b}&=\myvec{2\\2\\3} + \lambda \myvec{-1\\2\\1}=\myvec{2-\lambda\\2+2\lambda\\3+\lambda}
\end{align}
Now we know,
\begin{align}
	(\vec{a}+\lambda \vec{b})^{\top} \vec{c} = 0
\end{align}
Hence,
\begin{align}
	\myvec{2-\lambda & 2+2\lambda & 3+\lambda} \myvec{3\\1\\0} &= 0\\
	(2-\lambda)3 + 2+2\lambda &= 0\\
	6-3\lambda + 2 + 2\lambda &= 0\\
	\lambda &= 8
\end{align}

\end{document}

